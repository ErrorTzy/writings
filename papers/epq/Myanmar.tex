\documentclass[man,floatsintext]{apa7}
\usepackage{longtable}
\usepackage{indentfirst}
\usepackage{graphicx}
\usepackage[style=apa,backend=biber,isbn=false,url=false,eprint=false]{biblatex}
\DeclareLanguageMapping{english}{american-apa}
\addbibresource{../../.resources/biblatex.bib}
\title{\large{Using social media, what is the risk?} \\
\footnotesize{-- Booms, busts, and telecommunication fraud in Mynamar}}
\shorttitle{EXTENDED PROJECT QUALIFICATION}
\author{Xiaoyun Li}
\affiliation{
Centre Number: 94825 \\
Candidate Number: 1002 \\
Unit Number: P301 \\
EPQ June 2024
}
\usepackage{float}
\usepackage{todonotes}
\begin{document}
\maketitle
\tableofcontents
\newpage
\section{Introduction}

Telecommunications fraud poses a threat to Myanmar consumers, businesses, and telecommunications companies, causing economic losses, privacy, and reputation damage \autocite{graboskyTelecommunicationFraudDigital2001}. In this article, I will introduce the historical prospects of Myanmar and how the development of social networks promotes the development of telecommunications in Myanmar. Then, I will provide information on the definition of social networks, and the development of social networks and their impact on telecom fraud in Myanmar will prove to be important. Subsequently, several types of remote fraud and their shortcomings were explained. Finally, there are differences between different types of social networks and their differences in fraud. I am interested in this topic because one of my good friends has been deceived, so I always handle this type of scam very carefully. Research shows that nearly 10\% of the world's population is victims of telecommunications fraud. According to Whatmore's data, the proportion of victims of telecommunications fraud accounts for approximately 30\% of the total population in the country. What keeps Myanmar's fraud rate high? Now, let's delve deeper into the history of Myanmar.

\section{A Historical View on Myanmar}

The political situation in Myanmar has always been chaotic. Since gaining independence from British colonial rule in 1948, coups, uprisings, and revolutions have been occurring constantly. By the time this paper is being written, Myanmar has just gone through a political unrest, labeled as \textit{Operation 1027}. The Three Brotherhood Alliance, composed of the Arakan Army, the Myanmar National Democratic Alliance Army, and the Ta'ang National Liberation Army, rebelled against the junta and eventually seized control over Shan State \autocite{yunsunOperation1027Changing2024} This operation's impact extended beyond Shan State, inspiring offensives in other regions such as Kachin and Sagaing. In Kachin State, for instance, the Cochin Independence Army (KIA) attacked and overran strategic locations, leading:g to significant confrontations with the Myanmar armed forces \autocite{theinternationalinstituteforstrategicstudiesOperation1027Reshapes2023} However, the details of this operation are not our main interest here. What concerns us is the reason such uproar occurs.

The official account claims that the junta's oppression and the people's rage caused this coup \autocite{htetminlwinOperation1027End2023}. I believe this claim to be oversimplified and unsuccessful. No government intends to be evil, and even if there is a conflict of interest between the government and the people, the strife should not have lasted for such a long period. In the following section, I will propose an alternative reason for explaining such conflict. I argue that the problem stems from historical reasons by giving an account of a continuous struggle between different ethnic groups in Myanmar.

\subsection{Precolonial Period}

The nature of Southeast Asian politics itself has its own character, which \textcite{tambiahGalacticPolity2007} called "The Galactic Polity." This idea comes from "mandalas," originally reflecting how people thought the universe was organized (see figure \ref{mandala}). Accordingly, ancient Southeast Asian societies arranged their cities and kingdoms like mandalas, with the authority in the center and other important places around it. 

The consequence of such an arrangement is that the control is hierarchical but decentralized. Local rulers maintained a degree of autonomy within the overarching framework of the central authority. Another feature of the model is its flexibility. Instead of expanding and guarding borders, it is more like appending nodes to a tree. This reflects the fluid and adaptable nature of political authority and territorial control in Southeast Asia. \textcite{liebermanEthnicPoliticsEighteenthCentury1978} points out that such fluidity makes conflicts even deeper than ethnic identities like Peg, Ava, or Burma. The scope of this paper, however, would go no further than the classical account of ethnic conflict and leave Lieberman's theory for future analysis.

\begin{figure}[H]
        \centering
        \includegraphics[width=\textwidth]{mandala.png}
        \caption{\cite{MandalaBuddhistDeity1700}}
        \label{mandala}
\end{figure}

This is especially true for Myanmar. Historically, Myanmar hosts about seven major and many minor ethnic-linguistic groups. Almost all belongs to Tibet-Burman, Tai-Kadai, Austro-Asiatic, and Austronesian language group. Ethnic groups such as the Shan, Arakanese, and Mon have played significant roles in history \autocite[p.46-56]{aung-thwinHistoryMyanmarAncient2012}. Over 40 different sub-states have evolved within the Shan State. Royal Sawbwa, who has the power that rivals of the Burman kings, ruled this place. \autocite[18,54]{smithEthnicGroupsBurma1994}.

Geographically speaking, the most significant gap, both politically and culturally, is the gap between Upper Myanmar (the dry zone centered around the Irrawaddy River Valley) and Lower Myanmar (the delta region). The term in Burmese, anya (`upstream') and akye (`downstream'), already reflects that it is not only a geographical distinction but also a cultural one. Anyatha (`offspring of the upstream region') would consider akyitha(`offspring of the downstream region') as unintelligent and peculiar \autocite[43,142]{aung-thwinHistoryMyanmarAncient2012}.

The gap lies also economically. Lower Myanmar's historical development and culture were influenced significantly by international and regional trade in the Gulf of Muttama (Martaban) and the Bay of Bengal. This exposure to external influences made Lower Myanmar demographically and culturally diverse, adaptable, and less homogeneous ethno-linguistically. In contrast, Upper Myanmar's prosperous agriculture demands a stable society, and thus, it is more inwards and homogeneous. Before the colonization, Upper and Lower Myanmar are only united in The Second Pegu/Toungoo Dynasty (1539-1599), The Second Ava Dynasty (1597-1752), and the Konbaung Dynasty (1752-1886) \autocite[33,44,130]{aung-thwinHistoryMyanmarAncient2012}.

\subsection{British Colony Period}

The British entry into Myanmar was initially through trade. The East India Company enters northeastern Asia in the late 17th and early 18th centuries. Despite being latecomers compared to other European powers, the strategic importance of Myanmar grew for the British due to its proximity to India and the potential for trade routes. The actual process of colonization was through a series of conflicts known as the Anglo-Burmese Wars from 1824 to 1886. The First Anglo-Burmese War (1824-1826) was triggered by border disputes and resulted in the British control of territory, including the provinces of Tenasserim, Arakan, and Assam. By the Second Anglo-Burmese War (1852), Lower Myanmar was conquered. The final war, the Third Anglo-Burmese War in 1885, led to the entire colonization of Myanmar, and the original King, Thibaw, was exiled. \autocite[174-194]{aung-thwinHistoryMyanmarAncient2012}

The British put Aung San to power and adopted his strategy known as "divide and rule." This strategy. This strategy was intended to harmonize Myanmar's ethnic diversity through developing economy equally, and helping all ethnic groups to be independent. Yet it had its limitations, notably in the classification of ethnic groups and the degrees of autonomy offered to them. This "divide and rule" includes a dual division: `Ministerial Burma,' and the `Frontier Areas.' Basically, this division covers Upper and Lower Myanmar. The Frontier Areas were governed quite separately from Ministerial Burma. Ethnic minorities such as the Karen, Kachin, and Chin were preferred for recruitment into the colonial armed forces, forming ethnic regiments. Although some Burman historians accused the British of favoring minorities, the rule of colonization damages ethnic minority aspirations. It divides lands into different parts without consideration for nationality or economic development, leading to neglect and poverty. This division put the various ethnic groups on divergent paths toward political and economic development and posed problems for the unification of Myanmar in the future \autocite[18-23]{smithEthnicGroupsBurma1994}


\subsection{Postcolonial Period}

After the British left Myanmar, the country experienced significant transformations and challenges between 1942 and 1962. Initially, the competition among imperial powers during the Second World War, including Japan, China, and the United States, turned Myanmar into a battleground. Japanese controlled Myanmar along with the Burma Independence Army (BIA) for a short period of time. Then, the Japanese were repelled by BIA, and eventually, the independence of Myanmar was gained, and a parliamentary democracy was founded in 1948. \autocite[225-238]{aung-thwinHistoryMyanmarAncient2012}

Unfortunately, the newfound independence did not bring national unity. The country went into civil war, and the central authority no longer had the ability to maintain control over its territory, particularly in areas beyond the major urban centers. The army then became a dominant political force. Due to this instability, General Ne Win was invited to form a "Caretaker Government." This begins the military's direct involvement in governance, undermining the parliamentary democracy. Although the military government promised to return power to civilian authorities, the return to civilian rule in 1960 was short-lived. Ne Win's coup d'état in 1962 ended the experiment with parliamentary democracy and initiated decades of military rule. \autocite[238-244]{aung-thwinHistoryMyanmarAncient2012}  Since the illegitmacy of the military rule, many regions gathered their forces against the junta (see \textcite[34]{smithEthnicGroupsBurma1994}).

The junta transformed into a more representative system during the years, including introducing a new constitution and starting elections. \autocite[276-285]{aung-thwinHistoryMyanmarAncient2012}. Yet the coup in 2021 again reversed the democratization of the junta. Before the coup, Myanmar appeared on a path toward democracy, with significant electoral victories by the National League for Democracy (NLD) led by Aung San Suu Kyi. The military justified the coup by alleging widespread electoral fraud in the 2020 elections, while most observers do not accept this claim \autocite{ganesanMyanmar2021Military2023}.

\section{Defenitions and types of social media}

Understanding the basic history of Myanmar can explain the cause of the country's chaos, which is the ongoing colonial rule. Another reason is that due to the development of social media, there has been an increase in telecommunications fraud cases in Myanmar. Now, I will discuss the definition of social media in this article. The Oxford Dictionary defines social media as a broad online platform and technology that allows users to create, share, and share information, ideas, personal information, and other content within virtual communities and networks. The uniqueness of these platforms is that they allow users from all over the world to engage in social interaction, participation, and collaboration. Social media has changed the way people communicate and obtain information and even influenced social and cultural norms. It covers a wide range of services, including social media, Weibo platforms, content-sharing websites, etc. D. When discussing social media in an article, it is important to clearly define the specific types of social media platforms or their analysis, as each platform has its own unique characteristics, user groups, and impact on different aspects of society depending on the environment. Here are some of the most discussed types of social networks. On the one hand, there are social media platforms \autocite{rainieSocialNetworkingSites2011}. Facebook, LinkedIn, and WeChat platforms allow users to chat with friends, family, and professionals. Users can share updates, photos, and videos and participate in various forms of social interaction. In other words, this is a way of personal communication. Secondly, Weibo platforms such as Twitter and Tomboy are examples of users posting text messages, links, and media to exchange information, opinions, and content in compressed formats and promote the rapid dissemination of information. Unlike social media, Weibo is one of many communication platforms.

\section{The evolution of social media}

Let's take a look at how social media has developed on these different communication platforms. The development of social media is a way to change global communication and economic and social norms. This development can be viewed from different perspectives, including key dynamics, driving factors, impacts on different stakeholders, and regulatory frameworks \autocite{weberEmergenceEvolutionSocial2016}.

Early development, early start: The birth of social media can be traced back to six platforms, allowing users to create personal profiles and list friends. The second step is the rise of large platforms: LinkedIn launched in 2003, Facebook launched in 2004, YouTube launched in 2005, and Twitter launched in a new interactive online form. Along with it came a "mobile revolution": \autocite[531-558]{kapoorAdvancesSocialMedia2018}. The launch of smartphones and app stores at the end of the 21st century led to the explosive growth of mobile social media, such as Instagram in 2010 and Snapchat in 2011. \autocite{mcintyreEvolutionSocialMedia2014}. The final stage of evolution is the development of algorithms: over time, platforms gradually shift towards complex algorithms to personalize content, which affects the dissemination and consumption of information. \autocite{prellSocialNetworkAnalysis2012}. 

\section{Factors affecting the evolution of social media}

What is the driving force behind social media? First, "technological progress," that is, the innovation of Internet technology, mobile devices, and data analysis, has promoted the growth and performance of social network platforms. Secondly, changes in user expectations, such as instant access to information and social networks, may affect the development of mobile applications and features. The monetization models of advertising and data contribute to the sustainability of platforms in attracting and retaining users \autocite{kaunJoseVanDijck2014}.

\section{The effect of the evolution of social media on people}

How does social media affect people? With the development of social media, it will become easier to maintain relationships and connections with beginners from around the world. In addition, collecting and using personal data on the platform may also lead to serious privacy issues. It may also affect a person's mental health, including issues such as anxiety, depression, and addiction \autocite{simplerUnjustWebWe2012}. For businesses, social media has become an important marketing tool, enabling them to communicate more economically and effectively with their target audience. Direct communication on social media has changed service levels and customer engagement. The growth of influence has brought new opportunities and challenges to brands and advertising \autocite{deansEvolutionSocialMedia2018}. Similarly, social media will have a significant impact on market trends, consumer behavior, and various sectors of the economy as a whole. For example, in the recent COVID-19 epidemic, social media played a key role in disseminating information and emphasized the need for a platform to balance freedom of expression and public health responsibilities. The development of social networks is a continuous process influenced by technological, social, and economic factors. Its profound impact on individuals, businesses, and the global economy continues to grow, requiring appropriate regulatory action to address challenges and seize opportunities \autocite{kaneEvolutionaryImplicationsSocial2017}.

\section{The effect of the evolution of social media on telecommunication fraud}

Now, Let's take a look at the impact of social media on telecommunications fraud. Firstly, there are different types of telecommunications fraud. Just like the sound of fishing \autocite{zhangSocialMediaSecurity2016}. This is because scammers claim to come from legal institutions such as banks or government agencies and are searching for personal or financial information. In addition, "Internet fraud" includes different types of Internet fraud, such as email fraud. Scammers who send fraudulent emails need to "quickly scan" personal information or money. Limited demand and regulatory issues are factors. This has led to an increase in telecom fraud in Myanmar \autocite{canNewDirectionSocial2019}.

\section{How does telecommunication fraud happen}

After understanding the various types of telecommunications related to the development of social media, let's take a look at how scammers start their activities. Due to the widespread impact of social media on the market, fraudsters have many ways to engage in fraud on these platforms, among which common fraud methods and mechanisms mainly include anonymity, falsification of personal data, direct or indirect communication channels, and targeted advertising \autocite{hajiahmadImpactDigitalizationOccupational2020}. Some fraudsters choose to use anonymous means to impersonate others or forge identities to carry out fraudulent activities. Typical examples include spreading false information, impersonating others, or engaging in illegal activities in the name of others. Moreover, fraudsters often engage in phishing on social media through indirect or direct communication channels and even steal personal privacy information. In addition, these channels can also engage in online fraud through social engineering strategies, such as phishing emails, providing false work, or forging identities to obtain personal privacy or confidential information from unsuspecting victims as trustworthy individuals or organizations. Targeted advertising may cause privacy leaks, personal data is collected, and fraudsters can successfully deceive and connect with victims through social media, creating personalized and eye-catching information that suits the victims, reducing their vigilance, and thus engaging in fraudulent activities \autocite{baesensFraudAnalyticsUsing2015}.

\section{Methods and strategies of telecommunication fraud}

Next, I will introduce the methods and strategies of social media telecommunications fraud. There are many methods and strategies to deceive telecommunications on social media. For example, phishing scams allow fraudsters to impersonate legal entities through fake websites or publications and encourage users to provide sensitive information such as passwords, credit card information, or personal information. There are also incorrect promotions and discounts \autocite{choiVoicePhishingFraud2017}.
Fraudsters often use false promotions or discounts on social media to force users to disclose personal information or pay for non-existent products or services. Fraud and identity theft occurs when fraudsters impersonate individuals or businesses to gain trust and encourage users to share confidential information or engage in fraudulent transactions \autocite{chenPhishingScamsDetection2020}. However, fraudulent pyramid schemes can be promoted on social media, promising high investment returns but ultimately relying on recruiting new members to support the plan \autocite{bosleyDecisionmakingVulnerabilityPyramid2019}. Fraudsters can also create false identities on social media, establish romantic relationships with victims, and gain trust through economic or emotional exploitation.

\section{Role of social media in communication and connectivity}

It is necessary to understand the role of social media in communication platforms and how social media affects communication by studying the causal relationship between social media and telecommunications fraud. Firstly, the role of social media in communication and connectivity is a significant shift from traditional methods. To understand this change, it is necessary to compare the situation before and after the emergence of social media \autocite{haythornthwaiteSocialNetworksInternet2005}. Distribute before updating social networks. The traditional forms of communication are usually letters, telegrams, telephones, emails, and poster systems, which are the first form of social media. Letters and telegrams are very slow and take several days to weeks to be delivered. Although making a phone call is much more direct, it is often very expensive, especially for long-distance calls. Email can solve this problem by introducing digital convenience, but it lacks real-time interactivity. BBS allows users to participate in discussions, but there are limitations in scope and availability. Physical distance is the main obstacle \autocite{jensenSpeakingSystemSocial2017}. Remote communication has high costs and low frequencies. For many people, international long-distance calls are too expensive. In addition, due to cost and contact characteristics, the communication frequency is relatively low. Due to special circumstances or reasons, people can wait for remote communication. Therefore, traditional social media has its advantages and disadvantages in the past. From a positive perspective, more traditional communication methods are more cautious, and privacy issues are not as obvious. However, the disadvantage is that the speed is too slow, unpleasant, and often more expensive. Maintaining distance is more difficult.

\section{The influence of social media on communications}

Let's take a look at how social media influences communication methods. With the emergence of social media, more direct and diverse communication channels have emerged. Social media platforms such as Facebook, Twitter, Instagram, and WhatsApp have completely changed their way of communication by providing instant messaging, video calls, and the ability to share personal information and updates in real-time with a wide audience. Social media can also eliminate distance. Social media eliminates physical distance barriers, allowing people to stay in touch with friends and family from around the world without worrying about costs or time zones. Social media can also increase the frequency and timeliness of communication. Simple and cost-effective communication increases frequency. Nowadays, people can stay in touch every day or even for an hour to share the moments that have happened. By creating communities social media platforms can create interest-based communities across geographical boundaries. By obtaining information, social media has become the main source of news and information, although it also brings the challenge of misinformation. Therefore, distribution to social networks is more direct and profitable; it Strengthens communication, information collection, and remote communication skills. Data protection and train data security; Spread false information \autocite{kumarFalseInformationWeb2018}. The negative impact of excessive use or exposure to toxic substances on mental health. It also reduces personal interaction.

\section{Factors contributing to the amplification of fraud through social media}

By collecting basic information about the impact of social media on communication, factors that may exacerbate social media fraud can be introduced. Firstly, the anonymity on social media platforms makes it easy for scammers to create fake accounts and hide them behind the anonymity of social media platforms, making it difficult to track. When people engage in direct conversations, this is typically used on social media.\autocite{kumarFalseInformationWeb2018} Therefore, fraudsters are more likely to gain trust by creating false personal information. In addition, the speed of social media is also very fast, allowing scam plans to quickly reach a wider audience and enabling scammers to reach more victims in a shorter period of time. Fraudsters can leverage users' trust in social relationships to make them more vulnerable to fraudulent attacks spread through these networks. In addition, due to the limited reduction of social media platforms, their supervision and regulation are limited, leading to an increase in fraudulent behavior without sufficient supervision or intervention. What is the difference between social media platforms and Weibo social media platforms when telecommunications fraud occurs? Firstly, the likelihood of fraud is low because social media is a special type of personal communication platform with high levels of privacy and privacy. However, there are also possibilities for targeted advertising, fraud, phishing fraud, or identity theft. However, Weibo social media platforms offer a wider audience, which means that the information provided on these platforms can be viewed by a higher proportion of the audience. Therefore, anonymous fraud and false personal information require a wider audience.

\section{How can telecommunication frauds affect the victims of telecommunication fraud}

How do these factors affect victims of telecommunications fraud? Let's talk about the impact on the victims. Firstly, it may become a platform for the spread of fraudulent software on social media, thereby increasing public vulnerability. Victims of social media fraud may face social stigmatization or stigmatization. In the entire economy, victims of telecommunications fraud may suffer economic losses due to fraudulent programs promoted on social media platforms. Scammers can use social media to reach a wider audience, which may lead to more people being deceived and falling into financial difficulties. From a psychological perspective, telecommunications fraud can lead to betrayal, shame, and fear.

\section{Losses of the victims suffered from telecommunication fraud}

The losses suffered by individuals can be divided into economic losses, personal data losses, and reputation losses. People may become victims of fraudulent systems managed by social media, which can lead to direct economic losses such as unauthorized spending, false service payments, or investment fraud, which can cause significant losses to the middle class. In addition, social media scammers encourage people to provide sensitive personal or financial information, leading to identity theft, unauthorized access to accounts, and potentially causing new economic losses \autocite{huangCausesPreventionTelecommunication2018}.  In the event of reputational damage, victims of telecommunications fraud, especially those engaged in fraud through social media, may have an impact on their personal and professional relationships. When a company engages in fraud, consumer confidence in the company decreases, causing significant losses to the company. False social media advertisements claim that telecommunications companies and their brands or services operating in Myanmar have reduced customer loyalty and sales and lowered consumer confidence. In addition, the company may incur additional costs in investigating and combating fraudulent activities that affect its business, including implementing security measures, compensating affected customers, and managing losses. When a company's services are used for fraudulent activities on social media platforms, there is a risk of government censorship, fines, or litigation, which may affect economic sanctions and market position.

\section{How can social media affect the spread of telecommunication fraud}

With the increasing use of social media, the incidence of fraud may increase. The widespread use of social media will increase the frequency of telecommunications fraud, which can easily reach a wide audience and promote trust and connection on these platforms. Due to the widespread influence and criminal convictions of these platforms, victims of telecommunications fraud through social media platforms may also suffer economic losses, leading to fraud, personal false information, or identity theft. In addition, social media fraud damages people's trust in online interaction and transactions and may damage the reputation of telecommunications service providers and social media platforms. WhatsMore points out that frequent users of social media telecommunications fraud are more susceptible to identity theft, phishing attacks, and other fraudulent behaviors that may damage their trust in online services \autocite{ilzanUnderstandingPhenomenonRisks2023}. In addition, social media and telecommunications excuses also bring trust and security risks, including privacy and security issues, identity forgery and theft, and a lack of identity verification. Due to the severe lack of verification procedures, social media's handover and reliance on telecommunications can bring serious privacy and security issues. Personal information and privacy posted on social media platforms may be used as a means for fraudsters to forge or steal identities, thereby using your network of connections to commit fraud, which can also lead to trust crises. Moreover, personal information publicly available on social media platforms may be exploited and become part of fraudster's plans, making the development of fraud smoother.

\section{Conclusion}

Overall, the impact of social media on telecom fraud in Myanmar is significant and has a dual nature. On the one hand, social media platforms have widespread dissemination, so social media can also increase individual and collective awareness and prevention of telecommunications fraud. In addition, social media platforms can also help Myanmar combat fraudulent organizations. The next step is to disseminate information. Fraudsters can use social media to spread false information, attract potential victims to participate in fraud plans, and increase the number of fraud cases in the telecommunications industry. In addition, the community has also provided the answer: social media can quickly spread information, users can share their experiences, remind others of ongoing fraud, and support victims of telecommunications fraud.

\section{Future research suggestions}

For future research, he can conduct a comprehensive study to evaluate the direct impact of social media on the incidence of telecom fraud in Myanmar and analyze trends, user behavior, and response mechanisms to further evaluate the impact. It can also investigate whether existing rules are effective or whether new strategies are needed to regulate telecommunications services and combat fraudulent behavior on social media platforms. It would be beneficial to research innovative methods to increase social media users' awareness of the risks of telecom fraud in Myanmar, including interactive activities, targeted information dissemination, and collaboration with influential individuals through user training strategies. It can research and develop technology solutions such as artificial intelligence-based fraud detection tools or blockchain-based verification systems to improve fraud prevention and user security on social media platforms. By exploring these research areas, we can better understand the dynamics of telecommunications fraud on social media in Myanmar and develop more effective strategies to reduce risks and protect users from fraud.

\printbibliography{}
\end{document}
