\documentclass{article}
\usepackage{booktabs}
\usepackage[style=apa,backend=biber,doi=false,isbn=false,url=false,eprint=false]{biblatex}
\DeclareLanguageMapping{english}{american-apa}
\addbibresource{/home/scott/Desktop/WorkingFiles/VSCode/writings/.resources/biblatex.bib}

\begin{document}

\section{Northern Myanmar's Strife: A Historical View}

The political situation in northern Myanmar has long been chaotic. By the time this paper is being written, Myanmar has just went through a political unrest, called \textit{Operation 1027}. Three Brotherhood Alliance, composed of the Arakan Army, Myanmar National Democratic Alliance Army, and Ta’ang National Liberation Army, launched a significant rebellion in Shan State and other regions.\footcite{Operation1027Changing}

The operation reflects an increase in offensives, challenging narratives of fatigue within the revolutionary forces, and displaying a strong public support that has grown over time. This public support has been crucial, infusing the People’s Defense Forces (PDFs) and ethnic armed organizations (EAOs) with the resources and morale necessary to continue their fight against the junta​


I argue that the complexity of northern Myanmar is not contingent. It 

\begin{table}[htbp]
\centering

\begin{tabular}{@{}lllp{5.5cm}@{}}

\toprule
Name    & Population          & Main Religion(s)   & Main Armed Opposition group(s) \\
\midrule
Akha    & 100,000             & Animist            & --- \\
Burman  & 29,000,000          & Buddhist           & members of Democratic Alliance of Burma and CPB \\
Chin    & 750,000-1,500,000   & Christian, Animist & Chin National Front \\
Chinese & 400,000             & Buddhist, Taoist   & --- \\
Danu    & 70,000-100,000      & Buddhist, Animist  & --- \\
Indian  & 800,000             & Muslim, Hindu      & --- \\
Kachin  & 500,000-1,500,000   & Christian, Animist & Kachin Independence Organisation*, \\
        &                     &                    & New Democracy army* \\
Karen   & 2,650,000-7,000,000 & Buddhist, Christian & Karen National Union \\
Karenni & 100,000-200,000     & Christian, Animist & Karenni National Progressive Party \\
        &                     &                    & Karenni Nationalities People's Liberation Front* \\
Kayan   & 60,000-100,000      & Christian, Animist & Kayan New Land Party* \\
Kokang  & 70,000-150,000      & Buddhist, Taoist   & Myanmar National Democratic Alliance Army* \\
Lahu    & 170,000-250,000     & Animist, Christian & Lahu National Organisation \\
Mon     & 1,100,000-4,000,000 & Buddhist           & New Mon State Party \\
Naga    & 70,000-100,000      & Animist, Christian & National Socialist Council of Nagaland \\
Palaung & 300,000-400,000     & Buddhist           & Palaung State Liberation Front* \\
Pao     & 580,000-700,000     & Buddhist           & Pao National Organisation*, \\
        &                     &                    & Shan State Nationalities Liberation Organisation* \\
Rakhne  & 1,750,000-2,500,000 & Buddhist           & National Unity Front of Arakan \\
Rohingya& 690,000-1,400,000   & Muslim             & Arakan Rohingya Islamic Front, \\
        &                     &                    & Rohingya Solidarity Organisation \\
Shan    & 2,220,000-4,000,000 & Buddhist           & Mong Tai Army, Shan State Army* \\
Taovoyan& 500,000             & Buddhist           & Taovoyan Liberation Front \\
Wa      & 90,000-300,000      & Animist            & United Wa State Party* \\
\bottomrule
\end{tabular}
\caption{Population, Main Religion(s), and Main Armed Opposition Group(s) by Name}
\end{table}


\printbibliography{}
\end{document}
