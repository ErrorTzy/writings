\documentclass{article}
\usepackage{booktabs}
\usepackage{indentfirst}
\usepackage[style=apa,backend=biber,doi=false,isbn=false,url=false,eprint=false]{biblatex}
\DeclareLanguageMapping{english}{american-apa}
\addbibresource{/home/scott/Desktop/WorkingFiles/VSCode/writings/.resources/biblatex.bib}
\usepackage{todonotes}
\begin{document}

\section{Northern Myanmar's Strife: A Historical View}

The political situation in northern Myanmar has always been chaotic. By the time this paper is being written, Myanmar has just went through a political unrest, labelled as \textit{Operation 1027}. The Three Brotherhood Alliance, composed of the Arakan Army, Myanmar National Democratic Alliance Army, and Ta’ang National Liberation Army, rebelled against the junta and seized Shan State.\autocite{yunsunOperation1027Changing2024} This operation's impact even extended beyond Shan State, inspiring offensives in other regions such as Kachin and Sagaing. In Kachin State, for instance, the Kachin Independence Army (KIA) attacked and overran strategic locations, leading to significant confrontations with the Myanmar armed forces. \autocite{theinternationalinstituteforstrategicstudiesOperation1027Reshapes2023} The details of this operation is, however, not our main interest here. Despite the fact that Operation 1027 is believed to be related to cyber spam \todo{better expression}, what we are concerning here is what is the reason for which such uproar occurs. The official account, as well as the common opinion, claims that the junta's oppression and people's rage are the reasons for this coup. \autocite{htetminlwinOperation1027End2023} I believe this claim too be oversimplified.\todo{why? counterexample} In this section, I argue that the real reason for such conflict stem from historical reasons by give an account of a continuous struggle between different ethnic groups in Myanmar.

\subsection{The pre-colonial history of Myanmar}

The nature of southeast asian politics itself has its own character, which \textcite{tambiahGalacticPolity2007} called ``The Galactic Polity''. The concept of the mandala plays a key role in this ancient organizational blueprint. A mandala is a kind of map that shows how everything in the universe is connected, with the most important parts placed in the center. In the kingdoms of ancient Southeast Asia, the king's palace was the center of everything, much like the sun is the center of our solar system. Around the palace, like planets orbiting the sun, were the homes of nobles, temples, and the homes of common people, each placed according to their importance in society.~\autocite{tambiahGalacticPolity2007}

I argue that the complexity of northern Myanmar is not contingent. It 

\begin{table}[htbp]
\centering

\begin{tabular}{@{}lllp{5.5cm}@{}}

\toprule
Name    & Population          & Main Religion(s)   & Main Armed Opposition group(s) \\
\midrule
Akha    & 100,000             & Animist            & --- \\
Burman  & 29,000,000          & Buddhist           & members of Democratic Alliance of Burma and CPB \\
Chin    & 750,000-1,500,000   & Christian, Animist & Chin National Front \\
Chinese & 400,000             & Buddhist, Taoist   & --- \\
Danu    & 70,000-100,000      & Buddhist, Animist  & --- \\
Indian  & 800,000             & Muslim, Hindu      & --- \\
Kachin  & 500,000-1,500,000   & Christian, Animist & Kachin Independence Organisation*, \\
        &                     &                    & New Democracy army* \\
Karen   & 2,650,000-7,000,000 & Buddhist, Christian & Karen National Union \\
Karenni & 100,000-200,000     & Christian, Animist & Karenni National Progressive Party \\
        &                     &                    & Karenni Nationalities People's Liberation Front* \\
Kayan   & 60,000-100,000      & Christian, Animist & Kayan New Land Party* \\
Kokang  & 70,000-150,000      & Buddhist, Taoist   & Myanmar National Democratic Alliance Army* \\
Lahu    & 170,000-250,000     & Animist, Christian & Lahu National Organisation \\
Mon     & 1,100,000-4,000,000 & Buddhist           & New Mon State Party \\
Naga    & 70,000-100,000      & Animist, Christian & National Socialist Council of Nagaland \\
Palaung & 300,000-400,000     & Buddhist           & Palaung State Liberation Front* \\
Pao     & 580,000-700,000     & Buddhist           & Pao National Organisation*, \\
        &                     &                    & Shan State Nationalities Liberation Organisation* \\
Rakhne  & 1,750,000-2,500,000 & Buddhist           & National Unity Front of Arakan \\
Rohingya& 690,000-1,400,000   & Muslim             & Arakan Rohingya Islamic Front, \\
        &                     &                    & Rohingya Solidarity Organisation \\
Shan    & 2,220,000-4,000,000 & Buddhist           & Mong Tai Army, Shan State Army* \\
Taovoyan& 500,000             & Buddhist           & Taovoyan Liberation Front \\
Wa      & 90,000-300,000      & Animist            & United Wa State Party* \\
\bottomrule
\end{tabular}
\caption{Population, Main Religion(s), and Main Armed Opposition Group(s) by Name}
\end{table}


\printbibliography{}
\end{document}
