\documentclass{article}

\usepackage{todonotes}

\usepackage[style=apa,backend=biber,doi=false,isbn=false,url=false,eprint=false]{biblatex}
\DeclareLanguageMapping{english}{american-apa}
\addbibresource{.resources/biblatex.bib}

\usepackage{xeCJK}

\title{The genes and outer stimulation that can affect the performance of basketball}
\author{Some Author}

\begin{document}
\maketitle
\newpage
\tableofcontents
\newpage


\section{Introduction}


In the category of basketball, we used to believe that enhancing performance was all about physical training. It included team work, tactical analysis and many other things. But even though these factors definitely make basketball better, they have their own dangers too like getting hurt physically such as having a knee permanently injured, shoulder or finger or experiencing psychological damage due to being pushed too hard by the coach or bullied from teammates. This leads us to ask more questions about how genes and outside forces affect playing better in basketball. Understanding the size and strength of various external stimulations becomes even more important when we try to make plans for enhancing performance. The paper's goal is to understand how intrinsic genes and external stimulations work, comparing their importance and suggesting thoughtful methods for better basketball performance. 


\subsection{The intrinsic gene factors}
The game of basketball event lasts for 2 hours to 3 hours when playing 40-minute rules: four 10-minute matches in total and there are 5-minute breaks between each period. Minutes that the basketball player usually plays are about under 40 minutes in the actual game. The distance that the player runs is about five to seven kilometers and there are about 1000 movements in men and 650 in women. 1100 discrete movements occur with a player with altogether up to 217 high-intensity movements such as jumping, shuffling, and sprinting.\autocite[13-14]{raduScienceBasketball2018} As a result, basketball requires high level of endurance, explosive power and physical state for the players. Although these variables can be improved by the physical training or daily diet, your physical qualifications, which determined by the genes, can play an essential role in performing better in basketball.

According to the Cambridge dictionary, endurance is the ability to keep doing something difficult, unpleasant, or painful for a long time which requires a large amount of energy from the body cells to maintain the physical state. In this research, we are mainly discussing the endurance of the muscles. Including the ability to maintain the explosive power and muscle contract. People who have high endurance tend to stay in their best state for a long time. The game of basketball event lasts for 2 hours to 3 hours when playing 40-minute rules: four 10-minute matches in total and there are 5-minute breaks between each period. Minutes that the basketball player usually plays are about under 40 minutes in the actual game. The distance that the player runs is about five to seven kilometers and there are about 1000 movements in men and 650 in women. 1100 discrete movements occur with a player with altogether up to 217 high-intensity movements such as jumping, shuffling, and sprinting.\autocite[13-14]{raduScienceBasketball2018} As a result, the players may require good endurance to maintain their physical state, especially for the muscle cells. 

The reason why endurance can affect the physical state is that it affects the efficiency of aerobic respiration.\autocite{forsmanh.KoripalloilijanFyysinenHarjoittelu} Exercise and moving require constriction and relaxation of the muscle cells which require the energy from respiration. If the efficiency of aerobic respiration can be improved, the energy produce can increase, so that the muscle fibers can have the sufficient energy to contract and relax more regularly. Gens that are related to enzymes of energy metabolism, the raw material of the aerobic respiration and the storage of metabolites in muscles have been proved to play a role in affecting the aerobic respiration.\autocite{ahmetovGenesAthleticPerformance2016}

BDKRB gene, which is correlated with higher skeletal muscle metabolic efficiency and glucose uptake during exercise, helps to increase the amount of glucose in our body and helps to increase metabolic efficiency.\autocite{mareksawczukPolymorphismBradykininReceptor2013} \footnote{However, in this paper, the results have been tested that there's no significant relationship between the BDKRB gene and exercise. It can still be considered an influence factor due to its mechanism of action. There may also be errors in this study that cause the differences of the studies so this result should be further tested.} Another kind of gene ACE is part of the renin-angiotensin system and plays an important role in blood pressure regulation and maintenance of the water-electrolyte balance.\autocite{loefflerBiochemieUndPathobiochemie2019} Insertion of the DNA sequence of ACE is associated with increased endurance performance. The genes of the PPAR group also play an important role in the regulation of energy metabolism via mitochondria. Transcription factors of the PPAR group influence several genes involved in fat and carbohydrate metabolism and uptake of glucose into skeletal muscles \autocite{loefflerBiochemieUndPathobiochemie2019} and glucose is the raw material for aerobic respiration. These genes have shown their ability to increase the cell metabolism and the respiration, hoping to increase the endurance of the athletes.

Apart from the endurance, the efficiency of aerobic respiration is also considered as an important variable that can effect the basketball performance. Aerobic respiration is the general form for a series of reactions, which are usually described as four stages: glycolysis, link reactions, the Krebs cycle, and the electron transport chain. Aerobic respiration is the process for all the body cells to produce ATP, the energy, in order to perform the functions like contraction and dilation of the muscles. 

In the respiration, NAD and FAD, witch are both enzymes, play irreplaceable parts. They accept the hydrogen ion from the pyruvate and donate the hydrogen ion and electron in the electron transport chain so that the electrons and hydrogen ions can produce energy when passing the protein channels. As a result, if the hydrogen ions can be carried and released effectively in a short time, the rate of producing energy will increase. Slc12a8 gene synthesis protein that helps to produce NAD. The Slc12a8 proteins carry the NMD, which is the raw material of NAD, into the cell to produce NAD. \autocite{grozioSlc12a8NicotinamideMononucleotide2019} If the Sic12a8 genes are widely expressed, the amount of NAD might increase. As a result, the cells' ability to circulate the hydrogen ions will be more efficient, so the rate of aerobic respiration increases.

There are also some genes that have been scientifically proved to have an interplay with the sports performance. ACTN3 gene is a typical example. Scientists have proved that there's a strong relationship between the ACTN3 gene and sports performance. The presence of X and R alleles of ACTN3 has been reported to affect the sprinting and endurance abilities of elite athletes. About 92% of Olympic short-distance runners have been reported to have a gene with R section.\autocite{goelACTN3AthleteGene2007}

Last but not least, there are also some genes that influence the body qualification by affecting the muscle function, like C18ORF25. The study has shown that there's a strong correlation between the C18ORF25 gene and exercise capacity and muscle function. The C18ORF25 gene can improve and build up the skeletal muscle, which can be a huge benefit for athletes.\autocite{blazevPhosphoproteomicsThreeExercise2022}

Overall, most of the genetic reasons are inborn and cannot be changed easily. Although there are some ideas about changing the DNA base sequence of the genes like using the CRISPR/Cas9 to edit the DNA and improve the sports performance, it's not ethical and can pollute the gene pool, which cause a severe problem. What's more, the side effects from changing genes are serious. They may cause genetic diseases and strongly affect the offsprings. As a result, it's hard to perfect the basketball performance through genes, outer stimulations seem to have a more reliable way. 

\subsection{The outer stimulation}
Outer stimulation include artificial interruption and the natural condition. The natural condition includes the temperature, the whether, the altitude, the humidity and the basketball playground. They are strongly correlated to the mood and physical state of a player. However, these are hard to make a change to improve the basketball performance. The artificial interruption seems to be a more reliable way to change the basketball performance significantly. In this essay, we mainly focus on the auditory interference since the basketball players can be most directly affected by the sounds than the visual interruptions. \todo{()} Music has been proved to have effect on the basketball performance. In \textcite{mcleodConstructionMasculinityAfrican2009}'s essay about the Construction of Masculinity in African American Music And Sports, the idea of both music and sports such as basketball rely on rhythmic flow, polyrhythms and syncopations that disrupt and decenter expectations of audiences or opponents have been established. What's more, the physical embodiment of the beat also provides a synchronous reference point but also informs the musical content and ideas ``by infusing them with appropriate rhythmic vitality''. However, less experiments have been carried out to support the analysis. 

Another auditory interference is the yelling from spectators. In \textcite{eptingCheersVsJeers}'s studies, the relationship between the behavior of the spectators and the basketball performance. The result is quite surprising that the mean of the three conditions: cheers, jeers, and silence are the same. The author explains all the players are well trained so that the outer stimulation is not significant. However, other studies have shown clearly that audiences can impact the physiological variables of athletes (e.g., arousal, cardiac performance), as well as cognitive variables such as self-concept and perceptions of performance  (e.g., see \textcite{jonesAllWorldStage2007}). That produces conflict and investigating the relationship between audiences' behavior and the performance of the players is one of my purposes. Also, the extent of the influence between cheers and jeers, between the music stimulation or spectator's stimulation will be compared. In L. Kimberly Epting and Kristen N. Riggs' experiment,  the researchers compared the influence of cheers, jeers, and silence.\autocite{eptingCheersVsJeers} Nevertheless, the result shows that none of them can affect the athlete's behavior. The author gave the reason that the free throw is too easy for the players.

\section{The experiment}

\subsection{The experiment plan}
Initially, the experiment is planned to investigate the effectiveness of both music and the spectators on the performance of basketball. Unfortunately, due to the limitation from the school. It's hard to find the female audiences and most male audiences are friends or classmates of the basketball players. This may cause bias because it's hard to say whether the influence on basketball performance is caused by the yelling or simply the spectator himself. What's more, most of the spectators felt embarrassed and wared to cheer or bow their classmates, make the experiment even harder to operate.

To reduce the bias from people and make the experiment more reliable, only music is investigated at last. 

The aim of the essay now is to investigate the relationship between the tempo of music and the players' basketball performance

\subsection{The samples}
10 subjects are involved. They are all high school male students aged between 14 and 16 from Shanghai Baoshan World Foreign Language School and they are the members of school basketball team. All the subjects played basketball for at least once a week and all of them have trained basketball for at least 2 years. 6 of them have trained for other sports. 4 of them have trained badminton, 2 of them have trained table tennis and 2 people have practiced fencing. 

Before the experiment, they have asked to finish a questionnaire about their mood, their sleeping time and whether they have listened to music while playing basketball. They were told that the questionnaire was part of the daily training so that the demand characteristics is reduced. They were asked to scaled their mental state from 1-10 because the initial mental state may affect the performance of basketball \todo{()} and the average is 5.63. Sleeping time are also involved because people tend to behave differently depends on their sleeping time, and the mode time is 7 hours, which is within recommended sleeping hour according to the World Health Orgnisation. These data suggest that most of the subjects have a normal and healthy sleeping time with a peaceful mood. So the interruption caused by extremely good d and scaled by the number of
shoots, the goal number, the stealing number and the losing number. These
standards are chosen based on the NBA scouting report (“NBA 中国官方网站
| 球员资料,” n.d.). A behavior category is made based on these factors:
• Number of shoots: the number of throwing ball to the basket for each
player, including any ball that shoots in or not
• Goal number: the numor bad mood and the lack of sleeping time can be reduced. The reliability increase.

There are 3 experimenters in this study. One of them will stand on the second floor of the school gym to record the whole match with the camera. He should stand on the second floor because it's hard to be noticed by the players and can record the whole process. Another experimenter will stand beside the field as a spectator to record the data. He will also play the music during the process

A teacher is involved in the experiment who knew all the details of the experiment but he will not tell the players. He acts as a judge to regulate the composition. He will not interrupt the players in the experiment. 

\subsection{Procedure}
Before the experiment, 9 music of different tempos are chosen based on the range of BPM, which is the measurement of music tempo. 3 lento songs (ranged between 40-60 BPM), 3 moderato songs (98-112 BPM) and 3 Presto songs (168-200 BPM) are selected as the music we played.\autocite{HelloMusicTheory2022} Three music are selected for each to ensure there are always music of the same tempo during each 10 minutes' match.

The experiment was conducted in the school basketball gym. This experiment aimed to investigate the correlation between different tempo of music and the basketball performance. The 10 subjects are divided into two groups with 5 subjects in a group by drawing by lows. They first gathered in the basketball field, and their P.E. teacher will tell them it's just a normal full-court game, which consists of four 10-minute competitions. Next, the teacher will guide them to do a warm-up exercise for 5 minutes. The basketball match will not inform any other students so that less spectators are aware of the match and less spectators are present. During the match, any fouls will not stop the watch to ensue that the match can be finished on time and the afternoon class won't be affected.

In the first 10 minutes, no music will be involved. All the subjects are just having a normal competition. In the second 10 minutes, 3 presto music will be played in a medium loudness.

The first two 10 minutes are arranged after lunch and the rest competitions are arranged after the evening class to minimize the mood change of the players which may affect the result of the following experiments. During the afternoon break, some snacks are provided to maintain their mental state

Before the third match begin, the P.E. teacher will lead them to finish a 5-minutes' warming up exercise. Then, as soon as the match begins, 3 moderato music ( 108-120 BPM ) music will be played at a medium loudness.

In the last 10 minutes' match, the grave music is played. At the end of the match, the P.E. teacher will lead the players to do a stretching exercise and summarize their performance like usual to make sure the players are not aware of the experiment.

Thers's a five-minute break between each competition. The teacher will guide them as usual during break time.

After the experiment, three players are selected randomly to have an interview. First, there will be a apologize for the deception and misinforming the experiment. The subjects were told to inform other players in the team about the experiment. Next, I will ask them whether they have expected an experiment. Then, their feeling are asked for further information.

\subsection{Measurement}
The basketball performance are operationalized and scaled by the number of shoots, the goal number, the stealing number and the losing number. These standards are chosen based on the NBA scouting report \autocite{NBAZhongGuoGuanFangWangZhanQiuYuanZiLiao}. A behavior category is made based on these factors:

\begin{itemize}
\item Number of shoots: the number of throwing ball to the basket for each player, including any ball that shoots in or not
\item Goal number: the number of ball that shoots in the basket
\item Stealing number: the time that the player “steal” or grab the ball from the opponents
\item Lossing number: the time that the player lose their ball because of the attack by the opponents.
\end{itemize}




\section{Result}

\section{Discussion}

\newpage
\printbibliography{}
\end{document}