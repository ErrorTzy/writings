\documentclass{article}
\usepackage{xcolor}
\usepackage{soul}
\usepackage{todonotes}
\usepackage{indentfirst}
\usepackage{booktabs}
\usepackage{multirow}
\usepackage{tablefootnote}
\usepackage[style=apa,backend=biber,doi=false,isbn=false,url=false,eprint=false]{biblatex}
\addbibresource{../../.resources/My Library.bib}
% \usepackage{indentfirst}
\title{Effects of Music on Verbal Memory}

\author{some author}

\begin{document}
\maketitle{}


\section{Introduction}

We are capable of memorizing a lot of things without effort right from infancy. Yet we never really know how we managed to do that. Especially, the lack of understanding about memory lead us to suffer when we failed to recall what we should have memorized, say, when we are taking a test and having trouble recalling the right answer. 

In this paper, I first summarize the current researches on the common factors that affect memory. Then I \hl{  }. I close this paper by \hl{  }. I wish to show that \hl{  }.

\section{Literature Review}

First, we need to clarify what ``memory'' refers to. Intuitively, memories are what we recollect consciously, i.e.\ how to prove Pythagorean theorem. But some memories are also subconscious. For instance, nothing really pops up in our mind when we begin to ride. We just somehow control it. This indicates that some memories at work, but not at the conscious level. In psychology, this kind of subconscious memory is called \textit{implicit memory}, as opposed to \textit{explicit memory}.\footnote{Some scholars also use the ``declarative-nondeclarative'' dichotomy. As applied to humans, there is little difference between the ``declarative-nondeclarative'' and ``explicit-implicit'' taxonomy \autocite[p.480]{kolbIntroductionBrainBehavior2019}.} 

Another popular way of classification is by the temporal duration of memory, proposed by \textcite{atkinsonHumanMemoryProposed1968}. Ultra short-term memory is formed in the process of sensory registration, which lasts less than one second. Short-term memory lives up to thirty seconds. Long-term memory refers to memory that lasts indefinitely long period of time. 

The taxonomy is important because different types of memory corresponds to different neuro circuits. The prevailing view on how memory relates brain structure is illustrated in Table~\ref{BrainStructMemoryTypes}. 

\begin{table}[h]
\centering
\begin{tabular}{@{}ll@{}}
    \toprule
    Short-term      & Long-term \\
    \midrule
    \multirow{2}{*}{Prefrontal Lobe \tablefootnote{\cite{fusterCrossmodalCrosstemporalAssociation2000}}} & Explicit: Medial Temporal Lobe \tablefootnote{\cite{squireMedialTemporalLobe1991}}\\
                                                     & Implicit: Basal Ganglia \tablefootnote{\cite{foerdeRoleBasalGanglia2011}}\\
    \bottomrule
\end{tabular}
\caption{Brain Structures Corresponding to Memory Types}
\label{BrainStructMemoryTypes}
\end{table}

Now it follows that once the corresponding brain structure malfunctions, the related memory functions will also be compromised. 




Ageing is one of those. A large body of literature has shown that humans display losses in memory with age, but \hl{that} not all types of memory are affected equally. 

\hl{Much of} \todo{really?} the \hl{recent} \todo{recent?} neuroimaging work has been in the area of verbal memory, and a common finding in young adults is that they have increased activity in the left prefrontal cortex during \hl{encoding} \todo{encode what?} and in the right prefrontal cortex during \hl{retrieval} \todo{retrieve what?}. In contrast, older adults often have less activation of left frontal areas during encoding [3], but bilateral prefrontal activation during retrieval [4,5]. Older adults have also been found to have greater activity in left prefrontal cortex during performance of some nonverbal tests of retrieval [5], but not all [6].

The similarities between age-related declines in sensory and intellectual functioning are striking, and it has been reported that visual and auditory acuity, together, account for 93\% of the age-related variance in intelligence [8]. This also illustrates the correlation between intelligence and memory. However,age-related declines are  slight both in short-term memory span tasks, in which subjects repeat back a short string of words, letters or  numbers, and in many recognition memory tasks, in which  previously encountered events (e.g. words, sentences, pictures, faces) are represented along with new distractor items of a similar type.Decrements are also typically slight in implicit memory tasks (including priming) in which a stimulus that has been presented previously affects current behavior when presented again, often without the person realizing that the stimulus was encountered beforehand. 

Stress also plays an important role on memory. Enhanced memory for stressful or emotionally arousing events is a well-recognized,highly adaptive phenomenon they helps us to remember important information. Findings from experimental studies indicate that people have good recollection of where they were and what they were doing when they experienced an earthquake or witnessed an accident. Such memory enhancement is not limited to experiences that are unpleasant or aversive:pleasurable events also tend to be well remembered.[9] 

Moreover,the degree of attention can effect the memory.One main element of attention is the motivation. 

Another factor effecting memory -----nutrient deficiency also could not be ignored.Iron deficiency is one of the most common deficiency and it can influence memory.(Calson 2011).

Sleeping also is an significant element for memory.Sleep can improve primarily the consolidation of memory, while memory encoding and retrieval take place effectively during waking(Jan Born 2010).

Alcohol primarily impede the ability ti firm new long-term memories, leaving unimpaired previously established long-term memories and the ability to keep new information active in memory for brief periods.As the amount of alcohol consumed increases, so does the magnitude of the memory impairments.[10]


\printbibliography[title={Reference},heading=bibnumbered]{}
\end{document}