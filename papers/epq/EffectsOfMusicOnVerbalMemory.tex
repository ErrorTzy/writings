\documentclass{article}
\usepackage{xcolor}
\usepackage{soul}
\usepackage{todonotes}
\usepackage{indentfirst}
\usepackage{booktabs}
\usepackage{multirow}
\usepackage{tablefootnote}
\usepackage[style=apa,backend=biber,doi=false,isbn=false,url=false,eprint=false]{biblatex}
\addbibresource{../../.resources/My Library.bib}
% \usepackage{indentfirst}
\title{Effects of Music on Verbal Memory}

\author{some author}

\begin{document}
\maketitle{}


\section{Introduction}

We are capable of memorizing a lot of things without effort right from infancy. Yet we never really know how we managed to do that. Especially, the lack of understanding about memory lead us to suffer when we failed to recall what we should have memorized, say, when we are taking a test and having trouble recalling the right answer. 

In this paper, I first summarize the current researches on the common factors that affect memory. Then I \hl{  }. I close this paper by \hl{  }. I wish to show that \hl{  }.

\section{Literature Review}

First, we need to clarify what ``memory'' refers to. Intuitively, memories are what we recollect consciously, i.e.\ how to prove Pythagorean theorem. But some memories are also subconscious. For instance, nothing really pops up in our mind when we begin to ride. We just somehow control it. This indicates that some memories at work, but not at the conscious level. In psychology, this kind of subconscious memory is called \textit{implicit memory}, as opposed to \textit{explicit memory}.\footnote{Some scholars also use the ``declarative-nondeclarative'' dichotomy. As applied to humans, there is little difference between the ``declarative-nondeclarative'' and ``explicit-implicit'' taxonomy \autocite[p.480]{kolbIntroductionBrainBehavior2019}.} 

Another popular way of classification is by the temporal duration of memory, proposed by \textcite{atkinsonHumanMemoryProposed1968}. Ultra short-term memory is formed in the process of sensory registration, which lasts less than one second. Short-term memory lives up to thirty seconds. Long-term memory refers to memory that lasts indefinitely long period of time. 

The taxonomy is important because different types of memory corresponds to different neuro circuits. The prevailing view on how memory relates brain structure is illustrated in Table~\ref{BrainStructMemoryTypes}. This instantly follows that once the corresponding brain structure malfunctions, the related memory functions will also be compromised. 

\begin{table}[h]
\centering
\begin{tabular}{@{}ll@{}}
    \toprule
    Short-term      & Long-term \\
    \midrule
    \multirow{2}{*}{Prefrontal Lobe \tablefootnote{\cite{fusterCrossmodalCrosstemporalAssociation2000}}} & Explicit: Medial Temporal Lobe\tablefootnote{\cite{squireMedialTemporalLobe1991}}\\
                                                     & Implicit: Basal Ganglia \tablefootnote{\cite{foerdeRoleBasalGanglia2011}}\\
    \bottomrule
\end{tabular}
\caption{Brain Structures Corresponding to Memory Types}
\label{BrainStructMemoryTypes}
\end{table}

The first thing that directly comes to our mind that would harm the brain would probably be chemicals. For instance, when people are under stress, adrenal cortex releases \textit{glucocorticoids} due to our fight-or-flight response. This activates our amygdala and enhance our memory on the stress-related events \autocite{sapolskyStressAgingBrain1992}. Worse, evidence has shown that glucocorticoids would temporarily block memory retrieval \autocite[pp.119-144]{roozendaalStressMemoryAmygdala2009}, and long-term exposure to glucocorticoids kills hippocampal cells, which plays a critical role in explicit memory \autocite{sapolskyStressAgingBrain1992}. 

Alcohol is another chemical that is detrimental to memory. It suppresses the activity of pyramidal cells, the cells mainly found in medial temporal cortex, and impairs the formation of Hippocampal LTP\footnote{Long-Term Potentiation (LTP) mechanism, discovered by \textcite{blissLonglastingPotentiationSynaptic1973}, is considered as the mechanism of how brain learns new things}. Long-term alcoholism causes thiamine deficiency, which results the death of brain cells including medial regions of the thalamus and the mammillary bodies of the hypothalamus \autocite{martinRoleThiamineDeficiency2003}. This is known as \textit{Korsakoff Syndrome}. 

Iron deficiency can also affect memory given that iron participates in the metabolic of hippocampal cells. The deficiency of iron will cause abnormal hippocampal structure and plasticity in adults. Early life iron deficiency, on the other hand, impairs hippocampal development which in turn causes cognitive functions \autocite{frethamRoleIronLearning2011}.

Apart from chemicals, age is also an important factor, since old people tends to suffer from Alzheimer Disease and have slippery memory. This is usually caused by neuritic plaques, concentrated especially in temporal lobe areas related to memory \autocite{kolbIntroductionBrainBehavior2019}. In addition, older adults shows different patterns of brain activation from younger adults during explicit and implicit encoding and retrieval tasks \autocite{backmanBrainActivationYoung1997}.

Sleep also enhance memory the the process called \textit{consolidation}. Research has shown that both SWS and REM contributes to solidifying knowledge learnt in day \autocite{diekelmannMemoryFunctionSleep2010}.

\printbibliography{}
\end{document}