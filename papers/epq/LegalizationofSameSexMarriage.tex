\documentclass{article}
\usepackage{indentfirst}
\usepackage{graphicx}
\usepackage{float}
\usepackage{pgfplots}
\usepackage{filecontents}
\usepackage{todonotes}

\usepackage[style=apa,backend=biber,doi=false,isbn=false,url=false,eprint=false]{biblatex}
\DeclareLanguageMapping{english}{american-apa}
\addbibresource{/home/scott/Desktop/WorkingFiles/VSCode/writings/.resources/biblatex.bib}


\title{Legalization of Same-Sex Marriage}
\author{Some author}

\begin{filecontents}{data.csv}
Year,CumulativeCount
2001,1
2003,2
2005,4
2006,5
2008,6
2009,7
2010,10
2012,11
2013,15
2014,17
2015,22
2016,23
2017,26
2019,28
2020,30
2021,31
2022,32
\end{filecontents}
\pgfplotsset{width=10cm,compat=1.17}
\begin{document}
\maketitle
\newpage
\tableofcontents
\newpage
\section{Introduction}
In 1973, Kant wrote an essay to reject a common saying: ``That may be correct in theory, but it is of no use in practice''.~\autocite{kantCommonSayingThat2009} But here we are going to discuss a problem in the reversed order --- Something may be useful in practice, but is it correct in theory? 

Same-sex marriage has been a trend in recent years, as shown in figure~\ref{Cumu}. Up to 2022, 32 countries has allowed the marriage of gay and les. Especially, according to the International Monetary Fund (IMF), 41 countries and territories are listed as ``advanced economies'', in which 21, that is more than a half, has legalized same-sex marriage within 20 years. If we believe these developed countries are the future of most countries, we can say that, practically speaking, same-sex marriage is likely to be a trend in the future.


\begin{figure}[H]
    \centering
    \begin{tikzpicture}
        \begin{axis}[
            xlabel={Year},
            ylabel={Cumulative Number of Countries},
            xmin=2000, xmax=2023,
            ymin=0, ymax=35,
            xtick={2000, 2005, 2010, 2015, 2020},
            ytick={0, 5, 10, 15, 20, 25, 30, 35},
            legend pos=north west,
            ymajorgrids=true,
            grid style=dashed,
        ]
        
        \addplot[
            color=blue,
            mark=square,
            ]
            table[x=Year, y=CumulativeCount, col sep=comma] {data.csv};
            
        \end{axis}
    \end{tikzpicture}
        
    \caption{Cumulative Number of Countries Legalizing Same-Sex Marriage From 2000 to 2022~\autocite{perper32CountriesWorld2022}}
    \label{Cumu}
\end{figure}
    

Indeed, at the first glance, there seems no problem with this trend. If the function of marriage is the legal recognition of loving, then the exclusion of same-sex marriage seems arbitrary. No one seems to get hurt while the minorities are benefited. What a good deal! However, if we look closer, the problem we suggest rises. There are several counterarguments against same-sex marriage. In this paper, I will argue that \todo{...}

\section{Arguments against same-sex marriage}

According to \textcite{brakeMarriageDomesticPartnership2023}, same-sex marriage faces multiple challenges. In this section, I will \todo{...}

\subsection{Argument from alternative}
\label{alternative}

In the countries that do not promit same-sem marriage, for instance mainland China, sexual minorities can have the same rights as those who are married by signing a civil agreement. Countries like Italy has also passed the legal status of civil union.\autocite{povoledoItalyApprovesSameSex2016} Then why bother modifying marriage? 

It is problematic for at least three reasons. First, legalizing same-sex marriage pragmatically saves trouble. It is much easier to keep refining the definition of marriage itself than introducing an altogether new type of legal union. In most legal frameworks and policies, reference is made exclusively to marriage. Consider, for example, the case of a person lying incapacitated and facing major surgery. Normally, a hospital would accept only the closest relations or a spouse as the decision makers for that patient. Changing the definition of marriage would only leave existing laws in place, hence not require enormous changes in the legal code. Otherwise, where a new category of a relationship is to be put in place, then there would be a need for major amendments across many laws, as this only recognizes marriage as the form of the relationship. Thus, from a practical standpoint, amending the notion of marriage is a more efficient approach. 

Second, it is theoretically implausible. If marriage and civil union are exactly the same, despite the former one is heterosexual and the latter homosexual, and since we don't separate marriages by other factors, like race or class, why do so based on gender? Some may reply that, by definition, marriage means the heterosexual form. However, as we will see in the next section, it is implausible to say so. 

In addition, such distinction is compatible with social injustice. Logically speaking, distinguishing one thing from another does not necessarily follow that they are unequal. For instance, red is different from green, but they are equally colors. But how the government symbolically recognizes them indeed could affect how people look at them. As \textcite{mohrLongArcJustice2007} argues, reserving the term ``marriage'' only for heterosexuals while offering ``civil unions'' to same-sex couples serves to degrade gay men and lesbians by denying them access to an important social form that marks membership in full humanity. The reasoning is that marriage as a social form and ritual has profound cultural and symbolic significance beyond just the legal aspects. \Citeauthor{mohrLongArcJustice2007} believe that being denied access to ``marriage'' implies that same-sex relationships are inferior or less worthy. I do not think this argument can go this far. Nevertheless, providing an alternative is indeed compatible with discrimination and is less effective to reduce social injustice comparing with legalizing same-sex marriage.


\subsection{Argument from linguistic definition}

As mentioned in \ref{alternative}, some would argue that by linguistic definition, marriage means the heterosexual monogamous relationship. For instance, according to the American Heritage Dictionary, marriage is ``The legal union of a man and woman as husband and wife''.~\autocite{pickertAmericanHeritageDictionary2000} But we need to ask further: how do we know whether a definition of anything is true? If we ask what is a triangle, we may reply that it is a closed, three-sided polygon in a two-dimensional plane. We know this is true because we believe we have already known what a triangle is in a first place. Given the definition does not lead to any instance that we do not believe to be triangles, the definition is agreed to be true. What matters here is that, the definition of triangle does not come before our understanding of a triangle. Only have we some understanding of a triangle in the first place, can we determine if the definition is true or not. As of marriage, our understanding of what marriage should be precedes the normative definition of marriage. Therefore, we cannot answer what marriage should be by appealing to the existing definition.

\subsection{Argument from reproduction}
\label{reproduction}

Another classical argument against same-sex marriage is that the purpose of marriage is procreation. Since only a male and a female together can procreate, proponents of this view argue that only heterosexual marriage should be permitted. This argument may have its roots in natural law theory, but examining the foundations of natural law would far exceed the scope of this paper. Instead, I will focus on a practical problem with this argument. If procreation is the sole purpose of marriage, then infertile couples should not be allowed to marry, since they cannot have children. This logic would also extend to fertile couples who choose not to have children. Natural Lawyers, like \textcite{leeMarriageProcreationSameSex2008} defended that marriage is a multi-leveled personal union that is fulfilled by expanding into family through procreation, but remains good in itself even if procreation is not possible in a particular case. An infertile man and woman who marry still fulfill the conditions of marriage by 

\begin{enumerate}
\item {committing to the type of union that would be fulfilled by bearing and raising children together, and}
\item {performing the biological act of becoming ``one flesh'' through intercourse, even if other non-behavioral conditions prevent actual procreation.}
\end{enumerate}

Regarding condition A, an infertile man and woman can make this commitment in the same way that a fertile couple can because it is still a union that is naturally ordered towards procreation, even if procreation does not actually happen for them due to factors beyond their control. In contrast, as \Citeauthor{leeMarriageProcreationSameSex2008} argues, a same-sex relationship is necessarily disconnected from procreation. Regarding condition B, when an infertile man and woman engage in sexual intercourse, they still perform the kind of act that unites them biologically and makes them ``one flesh''. It is a \textit{type} of act that is towards reproduction. A same-sex couple, \Citeauthor{leeMarriageProcreationSameSex2008} claims, cannot unite biologically in this way through their sexual acts.

\Citeauthor{leeMarriageProcreationSameSex2008}'s arguments, I believe, are very bad. First, it is not at all clear why the infertile couple's sexual intercourse is still oriented to the bearing and raising of children in principle. Second, there is no reason to believe that bodily union is so important so that without which marriage could not take place. Even if so, it does not follow that the penile-vaginal sex is the only way to achieve bodily union.

In addition, procreation can happen outside marriage. It seems that, the only reason procreation should happen within marriage is that it is beneficial to child-rearing. Thus we come to the next section. 

\subsection{Argument from child-rearing}

Another argument is that, marriage includes the obligation of protecting and educating children. Parents tends to have stronger connection with their own child than adopted ones. \textcite{somervilleCaseSameSexMarriage2012} argues that a child has a right to be raised by their own biological mother and father whenever possible. Recognizing same-sex marriage would normalize situations where a child is intentionally deprived of either a mother or father from the beginning. In her view, while there can be legitimate exceptions where it is better for non-biological couples raise children, this should be regarded as a special case.

However I believe \Citeauthor{somervilleCaseSameSexMarriage2012}'s argument does not hold. Even if we do not pass the same-sex marriage law, it does not mean these couples will not adopt a child if they really want to. Legally normalizing or not, the phenomena always exist. Recognizing same-sex marriage does not necessarily follow that more children are being abandoned. Even if so, there is no reason to believe the children will have a better life if they are brought up by parents who abandoned them. Therefore, as \textcite{murphySameSexMarriageNot2011} suggests, \Citeauthor{somervilleCaseSameSexMarriage2012} is ``philosophically confused''.

\subsection{Argument from political neutrality}

Many people, especially those with religious beliefs, think homosexuality is morally wrong. But there are also many who believe it is morally acceptable. This creates a conflict or dilemma for society.~\textcite{jordanItWrongDiscriminate1995} argues that when there is this kind of moral disagreement, the government should try to find a middle ground that respects both sides as much as possible, rather than completely favoring one view over the other. He thinks banning gay marriage is a way to do this -- it sides with the ``homosexuality is wrong'' view in terms of official government recognition, but still allows homosexual relationships to exist privately. 

In addition, \Citeauthor{jordanItWrongDiscriminate1995} argues that if gay marriage was legal, it would force people who think homosexuality is immoral to act against their beliefs. For example, business owners would have to give health insurance to gay spouses of their employees, even if they believe gay marriage is wrong. Jordan believes it's unjust for the government to make people violate their deeply held moral or religious beliefs in this way, when there's no way for them to opt out. 

If Jordan's argument is correct, then we may need to bite some bullets. For instance, many of the same arguments were in fact made against interracial marriage -- that it violated the deeply held moral and religious convictions of a large number of people, and that legalizing it would force those opposed to act against their beliefs. However, we now believe that those arguments were misguided, and that banning interracial marriage was an unjust form of discrimination, regardless of how many people supported it based on their personal moral views. 

This, however, does not dismiss \Citeauthor{jordanItWrongDiscriminate1995}'s arguments. I believe his argument has reached the core of our debate and shows us the real problem. Should the state stay value-free, or impose value, namely justice, and lead citizens how to live their lives? \todo{...}


\subsection{Argument from slippery slope}

Among all the arguments mentioned in this section, the slippery slope argument is most different one. The rough idea of this type of argument is that, accepting a starting premise would lead to a chain of events that ultimately results in an extreme outcome. For instance, if society were to legalize same-sex marriage, some might argue that this would open the door to accepting other non-traditional forms of marriage, such as polygamy. Then if polygamy were to be permitted, it could be argued that there seems to be no good reason for prohibiting incestuous marriages. And if incestuous marriages were allowed, there seems to have no reasonable grounds to forbid marriages between humans and non-humans. 

The main idea behind the slippery slope argument is that once society deviates from the conventional understanding of marriage, any further expansion of that definition would appear arbitrary and without clear boundaries. Technically speaking, the slippery slope argument is not a single, specific argument but rather an ``argument schema'' -- a template or framework for constructing various arguments. In other words, it provides a structure that can be adapted to different situations and contexts to create a range of arguments based on the same underlying principle. In our case, it gradually deviates from the standard definition and finally reaches a very radical conclusion.

Before we try to reply this argument, I should point out why it iis the most distinct and potentially the most significant argument against same-sex marriage. The five arguments listed earlier can be addressed separately, each on its own merits. There does not appear to be a unified first principle that can structure all these replies into a systematic account. However, the slippery slope argument elevates the problem to a meta-level. Since it generates an infinite number of arguments, it is insufficient to simply attack arguments against same-sex marriage one by one.\footnote{It is worth noticing that the metaphor of a slippery slope is not entirely precise, since the argument is not linear. For example, even if we have enough reason to reject polygamy, it does not necessarily follow that we can reject incest.} To reply the challenge posed by the slippery slope argument, we must take a great effort to seek a systematic and comprehensive response.

\section{The slippery slope argument}

From the first glance, this argument seems reasonable. However, anyone who have studied logic would know that slippery slope counts as ``fallacy''. Why would that be so? Is the argument even valid? In this section, \todo{...}

\subsection{Validity of slippery slope}

At first glance, slippery slope arguments may seem fallacious. After all, they are often cited as an example of a logical fallacy. However, a closer examination reveals that the underlying logic of slippery slope arguments is more complex than it may initially appear.

To understand why slippery slope arguments can be valid, it is illuminating to consider the ancient philosophical puzzle known as the sorites paradox. The word "sorites" comes from the Greek word for ``heap''. The paradox, first formulated by the Megarian philosopher Eubulides in the 4th century BC, goes like this: Imagine a heap of sand. If you remove one grain of sand, you would still clearly have a heap. This seems to hold true no matter how many times you remove a single grain - removing one grain from a heap still leaves you with a heap. However, if you continue this process of removing a single grain over and over, you will eventually be left with a single grain of sand, which is clearly not a heap. So at what point did the heap stop being a heap? There seems to be no clear line.\autocite{hydeSoritesParadox2018}

This mirrors the logic of slippery slope arguments. The idea is that once you take that first small step from the status quo, there is no principled way to stop the slide down the slippery slope, because any stopping point you try to identify can be attacked using the same logic as the original deviation.  With same-sex marriage for instance, if you accept the step from the traditional definition of marriage, it becomes difficult to reject further expansions like polygamy, incestuous marriage, human-animal marriage, etc. Any line you try to draw between accepting same-sex marriage but not these other arrangements can be challenged. If you already deviated from the traditional definition of marriage between a man and a woman, what's your basis for rejecting a further deviation to allow marriage between three people, or between siblings, or between a human and a dog? You've already conceded the principle that marriage isn't limited to one man and one woman.

So in the same way that the logic of the sorites paradox makes it challenging to identify the precise point at which a heap stops being a heap, the slippery slope argument contends that once that first seemingly small concession is made, there is no clear, principled way to prevent sliding all the way down to a radical endpoint that initially seemed unthinkable. The strength of this style of argument is that it claims to identify an inexorable logic that will take hold once the first step is taken, even if advocates of that first step claim that they can hold the line there.

\subsection{The ultimate difficulty}

Philosophers have proposed various theories aiming to resolve the paradox, but each faces significant challenges. The epistemic theory \autocite{sorensenBlindspots1988,williamsonKnowledgeItsLimits2000} claims vague terms have precise but unknowable boundaries. However, this seems implausible given how vague words are actually used. It relies on boundaries being determined by an unknowable function of the word's unknowable pattern of use. That is to say, If ``heap'' had a precise meaning, why would it be unknowable to competent speakers? And how could the meaning be determined by an unknowable pattern of use? In addition, some of Williamson's arguments against other theories have also been accused of begging the question \autocite{wrightEpistemicConceptionVagueness2021}.

Supervaluationism \autocite{fineVaguenessTruthLogic1975,keefeTheoriesVagueness2000} says vagueness comes from our inability to decide on precise boundaries -- not because they don't exist, but because we haven't narrowed down the meaning enough. But it preserves the unintuitive conclusion that sharp boundaries exist, even if we can't locate them. It struggles to block the sorites paradox in both directions, since the arguments that no grains make a heap and that any number do both seem valid. Its semantics for quantifiers are non-standard, and its key notion of ``super-truth'' arguably fails to capture what we want from the concept of truth.

Many-valued logics and degree theories \autocite{goguenLogicInexactConcepts1969,zadehFuzzyLogicApproximate1975,smithVaguenessDegreesTruth2008} represent vagueness using numerical degrees of truth. A collection of grains might count as a heap to degree 0.6, say. As we remove grains, the degree of truth of "this is a heap" slowly decreases from 1 down to 0. This captures the seamless transition from heaps to non-heaps. But it still seems to impose precision in an arbitrary way. Why is something a heap to degree 0.6 rather than 0.5 or 0.7? What do those numbers really mean? Can we make sense of the notion of degrees of truth in the first place?

Contextualist accounts \autocite{kampParadoxHeap1981,soamesUnderstandingTruth1999} hold that the meaning of vague terms depends on the context of use. But this just shifts the paradox to how speakers use words, without resolving the underlying issue. Formulations using anaphora to fix context (``if that's a heap, then that is too...'') are not resolved by contextualism \autocite{stanleyContextInterestRelativity2003}. And some contextualist theories, like \textcite{shapiroVaguenessContext2006}, still rely on unintuitive non-classical logic.

The multi-range theory aims to keep standard logic by saying vague words can have multiple ranges of application, and a sentence can be true in one range and false in another. So ``1 million grains is a heap'' is true in some ranges and false in others. This blocks the paradox. But it implies an extreme form of relativism about truth that is hard to swallow. The notion of a range is also not fully clear.

In the end, the core difficulty in resolving the sorites is that any way of drawing a precise line between where a vague term applies and where it doesn't will seem arbitrary and fail to match the term's intuitive meaning and actual usage. Vague terms have seamless transition zones between clear cases and clear non-cases that resist the imposition of any precise boundary, however it is drawn.

\subsection{Solutions}

While the sorites paradox illuminates puzzling features of vague predicates in natural language, its relevance to socially constructed concepts like marriage is more limited. Marriage is not a natural kind with mind-independent boundaries, but an institution shaped by evolving social, legal, and cultural practices \autocite{coontzMarriageHistoryHow2006}. As such, its definition is amenable to being constructed through reasoning.

The criteria for sound marriage policy need not be arbitrary, but can appeal to deeper ethical principles. As Brake argues, the purpose of marriage law is a proper subject for moral and political philosophy, which can identify the core functions and social roles that justify access to marital status \autocite{brakeMinimizingMarriageMarriage2012}. 

\subsubsection{The nihilist solution}

Before we try to construct any systematic account of marriage, we need first ask if this thing called marriage exist at all. There is indeed a nihilist view about marriage law. According to this account, the state should get out of the marriage business altogether, leaving it to religions or private organizations \autocite{sunsteinPrivatizingMarriage2008,chartierPublicPracticePrivate2016}. However, I believe this confuses two senses of marriage: 

\begin{enumerate}
\item {As a personal ethical/religious commitment}
\item {As a public status involving legal rights and responsibilities}
\end{enumerate}

The marriage privatization argument assumes that the first sense is what really matters. But in debating public policy and law, it is the second sense that is at stake. And this is unavoidably a public matter. By attaching legal privileges and protections to some relationships and not others, the state expresses value judgments and creates differential social statuses. Abolishing civil marriage altogether fails to address the fundamental issue - as long as the state is in the business of incentivizing and regulating any type of close personal relationship (whether it's called marriage or something else), we must grapple with questions of equality and justice.

There is no neutral default. Even if the state rebrands civil marriage as civil unions or domestic partnerships, it still faces the substantive question of which adult personal relationships should be eligible for this status, and on what basis. Appeals to the sorites paradox cannot spare us the hard work of hashing out what types of relationship are valuable and deserving of public recognition and support. And this brings us back to the need for systematic moral reasoning about the nature and purpose of marriage as a public institution.

\printbibliography{}
\end{document}