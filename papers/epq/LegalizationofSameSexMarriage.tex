\documentclass{article}
\usepackage{indentfirst}
\usepackage{graphicx}
\usepackage{float}
\usepackage{pgfplots}
\usepackage{filecontents}

\usepackage[style=apa,backend=biber,doi=false,isbn=false,url=false,eprint=false]{biblatex}
\DeclareLanguageMapping{english}{american-apa}
\addbibresource{/home/scott/Desktop/WorkingFiles/VSCode/writings/.resources/biblatex.bib}


\title{Legalization of Same-Sex Marriage}
\author{Some author}

\begin{filecontents}{data.csv}
Year,CumulativeCount
2001,1
2003,2
2005,4
2006,5
2008,6
2009,7
2010,10
2012,11
2013,15
2014,17
2015,22
2016,23
2017,26
2019,28
2020,30
2021,31
2022,32
\end{filecontents}
\pgfplotsset{width=10cm,compat=1.17}
\begin{document}
\maketitle

\tableofcontents

\section{Introduction}
In 1973, Kant wrote an essay to reject a common saying: ``That may be correct in theory, but it is of no use in practice''.~\autocite{kantCommonSayingThat2009} But here we are going to discuss a problem in the reversed order --- Something may be useful in practice, but is it correct in theory? 

Same-sex marriage has been a trend in recent years, as shown in figure~\ref{Cumu}. Up to 2022, 32 countries has allowed the marriage of gay and les. Especially, according to the International Monetary Fund (IMF), 41 countries and territories are listed as ``advanced economies'', in which 21, that is more than a half, has legalized same-sex marriage within 20 years. If we believe these developed countries are the future of most countries, we can say that, practically speaking, same-sex marriage is likely to be a trend in the future.

Indeed, at the first glance, there seems no problem with this trend. If the function of marriage is the legal recognition of loving, then the exclusion of same-sex marriage seems arbitrary. No one seems to get hurt while the minorities are benefited. What a good deal! However, if we look closer, the problem we suggest rises. There are several counterarguments against same-sex marriage. In this paper, I will argue that 

\section{Arguments against same-sex marriage}




\begin{figure}[H]
\centering
\begin{tikzpicture}
    \begin{axis}[
        xlabel={Year},
        ylabel={Cumulative Number of Countries},
        xmin=2000, xmax=2023,
        ymin=0, ymax=35,
        xtick={2000, 2005, 2010, 2015, 2020},
        ytick={0, 5, 10, 15, 20, 25, 30, 35},
        legend pos=north west,
        ymajorgrids=true,
        grid style=dashed,
    ]
    
    \addplot[
        color=blue,
        mark=square,
        ]
        table[x=Year, y=CumulativeCount, col sep=comma] {data.csv};
        
    \end{axis}
\end{tikzpicture}
    
\caption{Cumulative Number of Countries Legalizing Same-Sex Marriage From 2000 to 2022~\autocite{perper32CountriesWorld2022}}
\label{Cumu}
\end{figure}


\printbibliography{}
\end{document}