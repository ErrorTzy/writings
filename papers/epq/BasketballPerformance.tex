\documentclass{article}
\usepackage[style=apa,backend=biber,doi=false,isbn=false,url=false,eprint=false]{biblatex}
\DeclareLanguageMapping{english}{american-apa}
\addbibresource{/home/scott/Desktop/WorkingFiles/VSCode/writings/.resources/biblatex.bib}

\usepackage{indentfirst}
\usepackage{xeCJK}
\usepackage{todonotes}
\usepackage{xcolor}
\usepackage{soul}
\title{Title Title}
\author{some author}

\begin{document}
\maketitle

\section{Factors of basketball performance}



Endurance is influenced by the genes that controls the synthesis of some specific proteins, while reflection is affected by both outer stimulation and genes.\todo{最好有引用} Playing basketball demands both stamina and strategic thinking. A typical basketball match, adhering to the 40-minute rule, spans between 2 to 3 hours, divided into four quarters of 10 minutes each, with 5-minute intervals between each quarter. Players generally spend less than 40 minutes on the court per game. During this time, a player covers a distance ranging from five to seven kilometers and performs approximately 1000 actions in men's games and 650 in women's games. \hl{(Männistö 2018.)} \todo{未知引用} A player engages in 1100 separate actions, including up to 217 high-intensity activities like jumps, shuffles, and sprints. \autocite[13-14]{raduScienceBasketball2018} As a result, the players may require good endurance to maintain their physical state, and a quick reflection for all these quick movements. 

In this section, I will mainly discuss how genes and outer stimulation affect the performance of basketball by analyzing the functions of some related genes and a correlational study to investigate the relationship between basketball performance and acoustic disturbance.

\subsection{Genetic Factors}

According to Cambridge Dictionary, endurance is the ability to keep doing something difficult, unpleasant, or painful for a long time which requires a large amount of energy from the body cells to maintain the physical state. In this research, we are mainly discussing the endurance of muscles,including the ability to maintain the explosive power and muscle contract. People who have high endurance tend to stay in their best state for a long time. 

On the physical level, endurance can be measured by the efficiency of aerobic respiration. \autocite{forsmanh.KoripalloilijanFyysinenHarjoittelu} Exercise and moving require constriction and relaxation of the muscle cells, and these actions need the energy from respiration. If the efficiency of aerobic respiration can be improved, the energy produce can increase, so that the muscle fibers can have the sufficient energy to contract and relax more regularly. Genes that are related to enzymes of energy metabolism, the raw material of the aerobic respiration and the storage of metabolites in muscles have been proved to play a role in affecting this aerobic respiration. \autocite{ahmetovGenesAthleticPerformance2016}  BDKRB gene, which is correlated with higher skeletal muscle metabolic efficiency and glucose uptake during exercise, helps to increase the amount of glucose in our body and helps to increase metabolic efficiency.\autocite{sawczukPolymorphismBradykininReceptor2013} \footnote{However, in this paper, the results have been tested that there’s no significant relationship between the BDKRB gene and exercise. It can still be considered an influence factor due to its mechanism of action. There are also errors \todo[inline]{如果有error需要指明是哪里;如果没有,不要写别人的研究有error} in this study that cause the differences of the studies so this result should be further tested.} According to \textcite{loefflerBiochemieUndPathobiochemie2019}, another kind of gene ACE is part of the renin-angiotensin system and plays an important role in blood pressure regulation and maintenance of the water-electrolyte balance. Insertion of the DNA sequence of ACE is associated with increased endurance performance. The genes of the PPAR group also play an important role in the regulation of energy metabolism via mitochondria. Transcription factors of the PPAR group influence several genes involved in fat and carbohydrate metabolism and uptake of glucose into skeletal muscles and glucose is the raw material for aerobic respiration.

\printbibliography{}
\end{document}