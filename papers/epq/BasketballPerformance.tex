\documentclass{article}
\usepackage[style=apa,backend=biber,doi=false,isbn=false,url=false,eprint=false]{biblatex}
\DeclareLanguageMapping{english}{american-apa}
\addbibresource{/home/scott/Desktop/WorkingFiles/VSCode/writings/.resources/biblatex.bib}

% packages for indentation
\usepackage{indentfirst}
\usepackage{xeCJK}
\usepackage{todonotes}

\title{Title Title}
\author{some author}

\begin{document}
\maketitle

\section{Factors of basketball performance}



Endurance is influenced by the genes that controls the synthesis of some specific proteins, while reflection is affected by both outer stimulation and genes.\todo{最好有引用} Basketball requires endurance and reflection. The game of basketball event lasts for 2 hours to 3 hours when playing 40-minute rules: four 10-minute matches in total and there are 5-minute breaks between each period. Minutes that the basketball player usually plays are about under 40 minutes in the actual game. The distance that the player runs is about five to seven kilometers and there are about 1000 movements in men and 650 in women. (Männistö 2018.) 1100 discrete movements occur with a player with altogether up to 217 high-intensity movements such as jumping, shuffling, and sprinting. (Vaquera and Radu 2019, 13-14.) As a result, the players may require good endurance to maintain their physical state, and a quick reflection for all these quick movements. 

In this section, I will mainly discuss how genes and outer stimulation affect the performance of basketball by analyzing the functions of some related genes and a correlational study to investigate the relationship between basketball performance and acoustic disturbance.

\subsection{Genetic Factors}

According to Cambridge Dictionary, endurance is the ability to keep doing something difficult, unpleasant, or painful for a long time which requires a large amount of energy from the body cells to maintain the physical state \todo{一般来说学术论文不引字典(文学/语言学除外)}. In this research, we are mainly discussing the endurance of muscles,including the ability to maintain the explosive power and muscle contract. People who have high endurance tend to stay in their best state for a long time. 

On the physical level, endurance can be measured by the efficiency of aerobic respiration. (Forsman et al. 2018, 34.) Exercise and moving require constriction and relaxation of the muscle cells which require the energy from respiration. If the efficiency of aerobic respiration can be improved, the energy produce can increase, so that the muscle fibers can have the sufficient energy to contract and relax more regularly. Gens that are related to enzymes of energy metabolism, the raw material of the aerobic respiration and the storage of metabolites in muscles have been proved to play a role in affecting the aerobic respiration. (Ahmetov et al., 2016).  BDKRB gene, which is correlated with higher skeletal muscle metabolic efficiency and glucose uptake during exercise, helps to increase the amount of glucose in our body and helps to increase metabolic efficiency. ( Human Biology, Vol. 85, No. 5 (October 1, 2013), pp. 741-756 ) However, in this paper, the results have been tested that there’s no significant relationship between the BDKRB gene and exercise. It can still be considered an influence factor due to its mechanism of action. There are also errors in this study that cause the differences of the studies so this result should be further tested. Another kind of gene ACE is part of the renin-angiotensin system and plays an important role in blood pressure regulation and maintenance of the water-electrolyte balance (Löffler and Petrides, 2013). Insertion of the DNA sequence of ACE is associated with increased endurance performance. The genes of the PPAR group also play an important role in the regulation of energy metabolism via mitochondria. Transcription factors of the PPAR group influence several genes involved in fat and carbohydrate metabolism and uptake of glucose into skeletal muscles (Löffler and Petrides, 2013) and glucose is the raw material for aerobic respiration. These genes have shown their ability to increase the cell metabolism and the respiration, hoping to increase the endurance of the athletes.

\printbibliography{}
\end{document}