\documentclass{article}
\usepackage{xcolor}
\usepackage{soul}
\usepackage{tikz}
\usepackage{forest}
\title{Notes}

\begin{document}
\maketitle

\section{Chapter 2}
\subsection{Plastic Patterns of Neural Organization}
Neural tissue has the capacity to change in response to the wold by changing how it is organized. People who suffer from cerebellar agenesis can slowly adapt; blind people can have enhanced auditory capacities. Especially, Learning means changes in neural circuits.

This is called \textbf{neuralplasticity}, which is a part of \textbf{phenotypic plasticity}, which means that the same gene can express itself differently in respond to the environment. 

\subsection{Functional Organization of the Nervous System}

\newpage
\begin{figure}[h]
    \begin{forest}
        for tree={
            align=center
        },
        [
            Nervous System
            [
                Central Nervous System (CNS)
                [
                    Brain
                ]
                [
                    Spinal cord
                ]
            ]
            [
                Peripheral Nervous System (PNS)
                [
                    Somatic\\Nervous System
                ]
                [
                    Autonomic\\Nervous System
                ]
                [
                    Enteric\\Nervous System
                ]
            ]
        ]
    \end{forest}
    \caption{Anatomical Organization}
\end{figure}

\begin{figure}[h]
    \begin{forest}
        for tree={align=center},[
            Nervous System
            [
                Central\\
                Nervous System
                [
                    Brain
                ]
                [
                    Spinal cord
                ]
            ]
            [
                Somatic\\
                Nervous System
                [
                    Cranial\\
                    Nerves
                ]
                [
                    Spinal\\
                    Nerves
                ]
            ]
            [
                Autonomic\\
                Nervous System
                [
                    Sympathetic\\
                    Division(arousing)
                ]
                [
                    Parasympathetic\\
                    Division(calming)
                ]
            ]
            [
                Enteric\\
                Nervous System
            ]
        ]
    \end{forest}
    \caption{Functional Organization}
\end{figure}



\end{document}