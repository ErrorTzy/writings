\documentclass{article}
% packages
\usepackage{indentfirst}
\usepackage{amsmath}

% info
\author{Tzy}
\title{Calculus I Notes}

\begin{document}
\maketitle{}
\section{Limit}

\subsection{Definition}

\textbf{Informal:} For \(\lim_{x \to c}\ f(x) = L\), you want to set f(x) \textit{as close as you want} to L by making x sufficiently close to c.

You tell me how close you want f(x) to be L; That is, you give me a positive range \(\epsilon\);

Then given this \(\epsilon\), then I will give you a \(\delta\) to show you that for any x within \(c \pm \delta\), f(x) will be within \(L \pm \epsilon\)
 
\textbf{Formal:} For \(\lim_{x \to c}\ f(x) = L\), Given \(\epsilon > 0\), we can always find \(\delta > 0\) such that for all x that \(|x - c| < \delta\), \(|f(x) - L| < \epsilon\) (That is, for all x in the range of \(\delta\), all the y are within the given range of \(\epsilon\))

\textbf{Example:} Prove:
\[
\lim_{x \to 5} f(x) = 10, f(x) = 
\begin{cases}
x, x = 5 \\
2x, x \neq 5
\end{cases}
\]

\textbf{Solution:} Now given the range of \(f(x) \pm \epsilon\), we want to show that our \(\delta\) can select x that has f(x) within that range. Notice that x and f(x) are all moving points. Now the given condition is that \(|f(x) - 10| < \epsilon\), and what we need to do is find the proper \(\delta\) that can restrict x, which in turn restricts f(x) so that \(|f(x) - 10| < \epsilon\). \(\delta\) restricts x through \(|x - 5| < \delta\). Since x is only approaching 5, \(x \neq 5\). Therefore we only need to consider \(f(x) = 2x\). Now we have equations:
\[
\begin{cases}
 |2x - 10| < \epsilon\\
 |x - 5| < \delta
\end{cases}
\]
We can solve this inequality by modifying the left hand side of the second inequality to the first, which is to multiply it by 2. Then we can see express \(\delta\) by \(\epsilon\), that is, we can make \(\delta = \frac{\epsilon}{2}\) so that \(|2x - 10| < \epsilon\)

\subsection{Properties}

We have \(\lim_{x \to c} f(x) = L\) and \(\lim_{x \to c} g(x) = M\)

\[
\lim_{x \to c} \left(f(x) + g(x)\right) = \lim_{x \to c} f(x) + \lim_{x \to c} g(x) = L + M
\]\[
\lim_{x \to c} \left(f(x) - g(x)\right) = \lim_{x \to c} f(x) - \lim_{x \to c} g(x) = L - M
\]
\[
\lim_{x \to c} f(x)g(x) = \lim_{x \to c} f(x) \cdot \lim_{x \to c} g(x) = L \cdot M
\]

\[
\lim_{x \to c} k \cdot f(x) = k \cdot \lim_{x \to c} f(x) = k \cdot L
\]

\[
\lim_{x \to c} \frac{f(x)}{g(x)} = \frac{\lim_{x \to c} f(x)}{\lim_{x \to c} g(x)} = \frac{L}{M}
\]
\[
  \lim_{x \to c} \left(f(x)\right)^{\frac{r}{s}} = \left(\lim_{x \to c} f(x)\right)^\frac{r}{s} = L^\frac{r}{s}
\]
\end{document}
