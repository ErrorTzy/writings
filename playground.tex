\documentclass[12pt]{article}

\usepackage[style=apa,backend=biber,doi=false,isbn=false,url=false,eprint=false]{biblatex}
\makeatletter
\let\blx@imc@addspace\addspace
\blx@regimcs{\addspace}
\makeatother
\DeclareLanguageMapping{english}{american-apa}
\addbibresource{/home/scott/Desktop/WorkingFiles/VSCode/writings/.resources/biblatex.bib}
\usepackage{musixtex}
\usepackage{subcaption}

\begin{document}
Jazz originated in the African American communities of New Orleans in the late 19th and early 20th centuries, evolving from a blend of ragtime, blues, and other musical styles into a unique form of expression that has influenced music worldwide.

Jazz's roots can be traced back to the African musical heritage and its fusion with European musical forms in the New World, leading to the creation of work songs, spirituals, ragtime, and the blues. This unique blend gave birth to jazz in New Orleans, where it was further developed and popularized by pioneers like Louis Armstrong, Duke Ellington, and Billie Holiday, among others. The city's vibrant music scene, characterized by a mix of cultures, played a crucial role in the genre's formation, with musicians from black, white, and Creole backgrounds contributing to its rich diversity \autocite{bindasHistoryJazz2000}.

\begin{figure}[hp!]
\centering



\begin{subfigure}{.3\textwidth}
    \centering
    \begin{music}
    \startextract{}
    \Notes\Dqbu gg\en
    \zendextract{}
    \end{music}
    \caption{Straight Rhythm}
    \label{noteA}
\end{subfigure}%
\begin{subfigure}{.3\textwidth}
    \centering
    \begin{music}
    \startextract{}
    \notes \uptuplet o20 \qu g \en \notesp \cu g \en
    \zendextract{}
    \end{music}
    \caption{Swing Rhythm}
    \label{noteC}
\end{subfigure}%
\begin{subfigure}{.4\textwidth}
    \centering
    \begin{music}
    \startextract{}
    \Notes\ibu0g0\qbp0g\tbbu0\tqh0g\en
    \zendextract{}
    \end{music}
    \caption{Swing Rhythm Emulated}
    \label{noteB}
\end{subfigure}

\caption{Explanation of swing rhythm}
\label{notes}
\end{figure}




\printbibliography{}


\end{document}
